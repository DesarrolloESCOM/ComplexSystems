\section{Sistemas complejos:  Definición e instituciones relacionadas en México}
  \paragraph{
  Un sistema complejo esta compuesto por partes interconectadas con vínculos que generan información adicional. Al existir interacción entre elementos, surgen propiedades nuevas que carecen de explicación a partir de los elementos inicialmente aislados.
  }
  \paragraph{
  Dichas propiedades son definidas como propiedades emergentes. Algunos ejemplos pueden ser:
  }
  \paragraph{
  Al existir una relación entre los componentes del sistema el mínimo cambio en algúna de las entidades que lo conforman se verá reflejado en todo el ecosistem al que este pertenece.
  }
  \paragraph{
  Todo sistema puede ser parte de otro sistema mayor que él, al que se le suele llamar supersistema.
  }
  \begin{itemize}
    \item El clima global.
    \item El cerebro humano.
    \item La organización social.
    \item El tránsito de una ciudad.
  \end{itemize}
  \paragraph{En México existen ciertas institucones que se dedican al estudio de sistemas no lineales y de la complejidad. A continuación se muestra un listado de éstos:}
  \begin{itemize}
    \item C3 - Centro de Ciencias de la Complejidad - UNAM.
    \item Maestría en Dinámica No Lineal y Sistemas Complejos - UACM.
    \item Laboratorio de Sistemas Complejos - UPIITA - IPN.
    \item Doctorado en Ingeniería en Sistemas - Sistemas Complejos - ESIME Zacatenco -IPN
    \item Doctorado en Física Aplicada y Teoría - CINVESTAV Mérida - IPN
    \item Línea de Investigación en Inteligencia Computacional y Optimización Avanzad - CINVESTAV Tamaulipas - IPN
    \item Línea de Investigación en Física de Sistemas Complejos - UAM Iztapalapa
  \end{itemize}
  \clearpage
