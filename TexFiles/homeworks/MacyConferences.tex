\section{Macy Conferences}
  \paragraph{
  Las Conferencias Macy fueron consideradas las reuniones más importantes de las metes de la época con el proposito de entender la conducta humana. Dieron como fruto las bases de la cibernética actual y avances significativos en la teoría de sistemas.
  }
  \paragraph{
  Estas conferencias tuvieron origen en Nueva York, Estados Unidos entre los años 1946 y 1953, fueron inicialmente iniciadas y organizadas por Warren McCulloch y la fundación Josiah Macy, Jr.Fundation. La unión de áreas tan diversas, es decir, desde fisiología hasta psicología, trajo consigo un avance y mezcla de ideas que en años posteriores resultaron ser revolucionarias y sentaron las bases de muchas áreas tanto técnicas como sociales hasta hoy en día.
  }
  \paragraph{
  Un total de diez conferencias se dieron cita, las primeras nueve fueron alojadas en el hotel Beekman en Nueva York, la última se encontró en Princeton, Nueva Jersey.
  }
  \paragraph{
  Los temas tomados en las conferencias fueron vastos y dieron frutos al dejar que el aislamiento disciplinario se rompiera generando así la únion de ramas de la ciencia que rara vez habían decidido compartir ideas. A continuación se muestran algunos tópicos vistos en éstas:
  }
  \begin{itemize}
    \item Redes neuronales simuladas emulando el cálculo de logica.
    \item Psicología en la infancia.
    \item Interpretación analógica vs interpretación digital de la mente.
    \item Lenguajes y la teoría de la información de Shannon.
    \item Información como semántica.
    \item Teoría de la decisión.
    \item La complejidad de los organismos como función de la información.
    \item Reconocimiento de formas y notas músicales en las redes neuronales.
  \end{itemize}
  \paragraph{Entre los científicos que asistieron, que rondan el número de 40, se mencionan los más destacados:}
  \begin{itemize}
    \item John von Neumann.
    \item Arturo Rosenblueth.
    \item Walter Pitts.
    \item Margaret Mead.
    \item Heinz von Foerster.
    \item Lawrence Kubie.
  \end{itemize}
  \clearpage
