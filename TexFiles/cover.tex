%%  COVER
\pagenumbering{Alph}
\begin{titlepage}
    \begin{center}
    \begin{tabular}{r c l}
    \includegraphics[scale=.20]{images/ipn} & \textbf{INSTITUTO POLIT\'ECNICO NACIONAL} & \includegraphics[scale=.20]{images/escom}\\
    & \textbf{ESCUELA SUPERIOR DE C\'OMPUTO}
    \end{tabular}
    \end{center}


    \vspace{1.5cm}
    \begin{center}


    \textbf{Computing Selected Topics:  Sistemas Complejos} \linebreak
    \large \textbf{``Simulación de un MicroMundo''} \linebreak

    \end{center}

    \vspace{1.5cm}

    \begin{center}

    \textbf{Reséndiz Arteaga Juan Alberto} \linebreak
    \end{center}

    \vspace{1.5cm}


    En el presente se encuentra el marco teórico, planteamiento, metodología, resultados y conclusiones del desarrollo de la simulación de un MicroMundo como sistema complejo, cuyo principal objetivo es la implementación de una simulación y determinación de los valores que permiten que el sistema simulado se comporte de forma estable. Esta simulación puede ser tomada como base para futuros trabajos relativos al área de Sistemas Complejos o bien como una referencia para análisis estadísticos relativos al comportamiento individuos con acciones y recursos limitados dentro de un ambiente controlado. \linebreak


    \vspace{1.5cm}

    \begin{center}

    \end{center}
\end{titlepage}
\pagenumbering{arabic}
