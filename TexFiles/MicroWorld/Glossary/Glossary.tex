\newpage
\section*{Glosario}
\begin{itemize}  
  \item \textbf{Políglota}: En área informática, que tiene soporte o comprende varios lenguajes de programación.  
  \item \textbf{Simular}: Representar algo fingiendo o imitando lo qué no es.
  \item \textbf{Emular}: Imitar las acciones de otro procurando igualarlas o incluso excederlas.
  \item \textbf{Trófica}: Relacionado a la nutrición. Relativo a la cadena alimentícia.
  \item \textbf{Contrato}: Acuerdo bilateral por el cual una de las partes se obliga a desarrollar un programa de ordenador, normalmente partiendo de un programa estandar, que se ajuste a las necesidades y objetivos de la otra parte.
  \item{\textbf{Multiparadigma}: Es aquel que porta más de un paradígma de programación, por ejemplo, aquellos que implementan conceptos de programación funcional, estuctural y orientado a objetos.}
  \item{\textbf{TDD}: Test-Driven Development, técnica de desarrollo que basa su conducta en la creación de pruebas previo a la creación del código que implementará la lógica de negocio.}
  \item{\textbf{Refactor}: Técnica de desarrollo de software donde se mejora la estructura y ejecución de un trozo de código sin alteral la funcionalidad final de éste.}
  \item{\textbf{Trait}: Concepto usado en programación orientada a objetos, éste representa una colección de métodos y propiedades que pueden ser usadas para extender la funcionalidad de alguna clase.}
  \item{\textbf{Closure}: Se le conoce a toda función evaluada en un entorno que contiene una o más variables dependientes de otro entorno, es decir, se puede definir una función dentro de otra función donde se la última definida puede hacer uso de los recursos de la primera.}
  \item{\textbf{Lambdas}: Proveniente del Cálculo Lambda. Hace referencia a una función anónima en programación. Éstas funciones nos permiten generar funciones que pueden ser utilizadas dentro de un contexto y no volver a usarse, evitando la generación de código innecesario.}
\end{itemize}
\addcontentsline{toc}{chapter}{Glosario}