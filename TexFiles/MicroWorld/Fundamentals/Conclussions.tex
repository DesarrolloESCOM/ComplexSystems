\chapter{Conclusiones}
  El análisis de los ecosistemas tiende a ser más sencillo tomando como base un sistema complejo en lugar de un enfoque simbólico (Inteligencia Artificial), ya que nos muestra la interacción sin llevar un control de forma directa sobre éstos.

  Los resultados arrojados, tomando en cuenta la gráficas y el video de demostración, permiten la visualización de como los cuerpos de agua son algo vital para la supervivencia de las especies, además del proceso de comunicación para que las grandes manadas puedan formarse.

  En el video demostrativo es claro como de comenzar con valores totalmente aleatorios, todos los animales comienzan a realizar la búsqueda de elementos que los hagan permanecer con vida. La reproducción, que a pesar de ser de forma asexual, nos brinda un panorama donde la creación de nuevos individuos enriquece a las manadas y genera más ojos para que puedan lograr obtener información (agua, tierra, plantas y depredadores).

  Considerando lo visto en clase y lo aprendido por investigaciones personales, puedo concluir que el enfoque de los sistemas complejos para la solución de problemas de la vida cotidiana es algo más que práctico, ya que nos permite modelar sistemas muy diversos para su posterior análisis. Sin embargo, es necesario contar con ciertos conocimientos y habilidades para poder generar procesos de simulación precisos y concretos.

  El problema más grande es el proceso del modelado del problema, ya que entre más estados y acciones tengan los automatas, mejores resultados se podrán obtener a costa de un gran poder de cómputo. Cabe destacar que no se puede implementar del todo el concepto de un autómata celular, ya que se carece de la cantidad de procesos (Hilos o Workers) únicos para poder simular la paralelización de éstos.

  A título personal, me quedo conforme con lo aprendido en la asignatura y realmente me encuentro interesado en un futuro volver adentrarme en un área tan vasta.