%
\newpage
\chapter{Introducción}
  \section{Sistemas Complejos}
      \paragraph{Un sistema complejo es definido tipicamente como un sistema que es \"más que la suma de sus partes\", es decir, mientras que de forma individual los elementos del sistema pueden ser muy sencillos y fáciles de analizar, el comportamiento del sistema como un todo es altamente complejo y difícil de predecir.\cite{3}}
      \paragraph{Un sistema complejo se encuentra formado por un número elevado de componentes elementales (autómatas), que interactúan de forma local entre ellos y con el entorno que los rodea. Son sistemas cuya evolución es muy complicada de predecir ya que aparecen comportamientos (comportamiento emergente) que es difícil de predecir considerando los parámetros iniciales.}
      \paragraph{Las interacciones entre los individuos no suele ser lineal, es decir, existen proceso de retroalimentación, comunicación o inhibición en los mismos. Pueden variar su estructura con el tiempo, es decir, la creación de nuevos individuos, nuevos enlaces; como un sistema dinámico. \cite{1}}
      \paragraph{Una de las premisas más grandes es la que menciona que comportamientos simples suelen dar lugar a comportamientos macroscópicos complejos, como se pudo ver al realizar el análisis de los atractores del juego de la vida de John Conway.\cite{2}}
    \subsection{Características de los sistemas complejos}
      \paragraph{Una de las principales características de los sistemas complejos es la convivencia y relación entre los elementos con el sistema y consigo mismos, por ejemplo, un agente o automáta que tiene una visibilidad limitada de sus sitema. Un claro ejemplo es la vecindad de Moore utilizada mucho en el análisis de automatas celulares.\cite{4}}
      \paragraph{Otro factor a tomar en cuenta como principio de los sistemas complejos es que todas la unidades (automatas o elementos) son considerados como trabajos en paralelo, es decir, podrían ser vistos como computadoras independientes y cuyo procesamiento no depende uno del otro.}
      \paragraph{Los sistemas complejos como un todo muestran fenómenos emergentes, fuera de las interacciones entre las unidades simples, emerge un comportamiento complejo, patrones o inteligencia. Es bien sabido que los ejemplos existen, por ejemplo las colonias de hormigas, patrones de migración, terremotos, copos de nueve, etc.}
      \paragraph{Finalmente, se pueden identificar tres aspectos de los sistemas complejos, es importante mencionar que puede ser una diversa mezcla de éstas y no todos los sistemas complejos las tienen.}
      \paragraph{\textbf{No lineal}: Este aspecto de los sistemas complejos suele ser referido como \"el efecto mariposa\", acuñado al matemático y meteorólogo Edward Norto Lorenz, pionero en el estudio de la teoría del caos. Se le llama \"no lineal\" por qué no hay una relación donde un ligero cambio pueda ser visualizado dada una cierta función, es decir, pequeños cambios implican nuevos resultados. Los sistemas no lineales son un superconjunto de los sistemas caóticos.}
      \paragraph{\textbf{Competición y cooperación}: Una de las cosas que suelen ser visualizadas en un sistema complejo es la presencia de competición y cooperación entre los elementos. Esto puede ser visualizado durante el análisis de ciertos procesos naturales, por ejemplo, el sistema inmunológico; donde se puede ver a los glóbulos blancos (Linfocítos) distribuirse, comunicarse y cooperar para la erradicación de posibles invasores al sistema. No todod los sistemas complejos pueden mostrar ese tipo de comportamiento, por ejemplo, un análisis climatológico.}
      \paragraph{\textbf{Estocásticos}: Esta propiedad nos habla del azar o bien de la facultad de tomar o no cierta acción o decisión. El comportamiento de un sistema complejo puede ser reflejado mediante proceso aleatorios dentro de sus movimientos o interacciones dentro del sistema y con sus vecinos.}
      \paragraph{\textbf{Retroalimentación}: Los sistemas complejos suelen contener retroalimentación donde la salida de un sistema se convierte en la entrada del mismo generando resultados distintos. Esta es una de las principales causas de que los sistemas complejos no sean lineales.\cite{3} \cite{5}}